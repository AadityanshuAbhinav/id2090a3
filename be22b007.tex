\section{BE22B007}

Name - Aditya Suhas Tamras \\
Roll No. - be22b007 \\
GitHub UserID - AdityaTamras \\ 

\title{\textbf{\underline{1st Law of Thermodynamics}}}

\begin{equation}
\delta U = \delta Q + \delta W
\end{equation}

Here $\delta$ U = Change in internal energy of the system \\
$\delta$ Q = Heat exchanged by the system with the surrounding \\
$\delta$ W = Work done by/on the system \\


Many experiments have shown that the heat effect q and
the work effect w vary for different paths between a given
initial state and a given final state of a reactive system, yet the
sum of q and w is always independent of the path between
the same initial and final states. In thermodynamics, we
presume these remarkable observations are always true and
generalize them into the above first law of thermodynamics. Here, the internal energy is independent of the path i.e is a state function whereas the work done and the heat exchanged are path functions i.e are dependent on the path taken by the system to reach from initial state to final state. 

The essence of the first law is that the sum of the heat effect and the work effect is the same regardless of the path between a given initial state
and a given final state. The law is in accordance with the principle of conservation of energy. The law gets slightly modified depending upon the sign convention used for the parameters (here, work done by the system is taken to be negative, heat escaping the system to be positive and increase in the internal energy as positive).

\footnote{
Eric A. Gislason and Norman C Craig. First law of thermodynamics; irreversible
and reversible processes. Journal of Chemical Education, 79:193–200, 2002
\url{https://pubs.acs.org/doi/pdf/10.1021/ed079p193}
}
