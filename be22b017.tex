\section{BE22B017}
Name      : Aditya Raj \\
Github ID :PantherUnica\\
\\
YDSE uses two coherent sources of light placed at a small distance apart.
it helped in explaining the wave theory of light .\\

Conditions for obtaining the equation:     
\begin{itemize}
\item slit separation and the screen distance should be kept unchanged and          should be large enough .

\item the two lights are assumed to be parallel.

\item slit distance/wavelength often d is a fraction millimetre and wavelength is a fraction of micrometre for visible lights.
\end{itemize}

The distance of the nth  bright fringe from the centre is:

\begin{equation}
    x_n = \frac{n\lambda D}{d} 
\end{equation}
  
The distance of the nth  dark fringe from the centre is:
\begin{equation}
    x_n =  \frac{(2n+1) \lambda D}{2d} 
\end{equation}

fringe width:
\begin{equation}
    \frac{\lambda D}{d} 
\end{equation}
in the equation the variables:
$D $ = distance between the two screen
$d $ = distance between the two slits 
$ \lambda $ = wavelength 
$x_n $= distance \\
\footnote{
Santu Nath and Pintu Mandal. Physics education shape of interference fringes in young’s double slit experiment. 01 2020
}
