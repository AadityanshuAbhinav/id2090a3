\section*{BE22B020}
Name : Arjun Aravindan \\
Github ID : arjun052005\\
\\
This article describes the derivation of terminal velocity from stokes law, which in itself was derived from experiments done by George Gabriel stokes.In this derivation of terminal 
The force of viscosity on a small sphere moving through a viscous fluid is given by:

\[F = 6 \pi \eta rv\]
\small{\begin{itemize}

\item $F$ is the frictional force acting on the interface between the fluid and the particle known as Stokes' drag
\item $\eta$ is the dynamic viscosity
\item $R$ is the radius of the spherical object
\item $v$ is the flow velocity relative to the object.

\end{itemize}}

In this derivation of terminal velocity there are certain assumptions, which include:
\small{\begin{itemize}

\item  The particle must be treated as solid, smooth, and spherical in shape.
\item  The density of the particle must be uniform.
\item  The radius of the spherical particle must be greater than 0.001 mm compared to fluid molecules so that the thermal (Brownian) motion of the fluid does not affect the particle.
\item Particles should not interfere with each other during the fall.
\item The flow should be laminar.

\end{itemize}}

\textbf{Derivation of terminal velocity}\\

When a small spherical object moves in a Newtonian fluid and reaches its terminal velocity, the total external force acting on it becomes zero. Let the density of sphere be $\rho$ and density of fluid against which it moves be $\rho_0$  

\[F_T + F_V = W\]
\[\frac{4}{3}  \pi r^3 \rho_0 g + 6 \pi \eta rv = \frac{4}{3} \pi r^3 \rho g\]
\[6 \pi \eta rv = \frac{4}{3} \pi r^3 ( \rho - \rho_0 )g\]
\[v = \frac{1}{2 \eta } r^2 ( \rho - \rho_0 )g\]

\footnote{G. G. Stokes, “On the Effect of Internal Friction of Fluids on the Motion of Pendulums,” Transaction of the Cambridge Philosophical Society, Vol. 9, Part 2, 1851, pp. 8-106.}
