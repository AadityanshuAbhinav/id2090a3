
\section{me22b002}

\footnote{
Reference from: \\
https://www.edn.com/einstein-paper-outlines-mass-energy-equivalence-november-21-1905/}Albert Einstein’s paper “Does the Inertia of a Body Depend Upon Its Energy Content?” was published in the journal “Annalen der Physik” on November 21, 1905.

The paper revealed the relationship between energy and mass that would eventually lead to the famous mass-energy equivalence formula 

\begin{equation}
    E = mc^2
    \label{main_eqn}
\end{equation}

\textit{\begin{center}
                \textbf{where E is body's kinetic energy}
                \textbf{m is the body's increased relativistic mass}
                \textbf{c is the speed of light}
        \end{center}}

\footnote{https://earthsky.org/human-world/einsteins-most-famous-equation-emc2/}It sounds simple. And its simplicity might obscure the genius, and the innovation of thought, required of Einstein to express it all so elegantly. Mass and energy are interchangeable. Plus, a small amount of mass can equal a large amount of energy. After all, the speed of light is a huge number (186,000 miles per second or 300,000 km/s). And, in Einstein’s famous equation, that huge number is squared. So it doesn’t take much mathematical skill to see that a tiny mass can equal big energy. \\


Equation \ref{main_eqn} explains why the sun and other stars shine. In their interiors, atoms (mass) fuse together, creating the tremendous energy of the sun as described by Einstein’s famous equation.

\begin{itemize}
    \item Name: Taher S A
    \item Github User ID: Taher-SA
\end{itemize}

