\section{ME22B007}
Name:Jaanav Mathavan\\
UserName:Jaanav-Mathavan\\
\subsection{Introducton}
Euler's formula is one of the most well known revolutionary equation that has given us a way to approach imaginary numbers.>
\subsection{THE BOOK: Introduction to Analysis of the Infinite}
Leonhard Euler (1703-1783) is commonly regarded as one of the greatest mathematicians of all time, and the content of this >
Many of the notations essential to mathematics are provided by Euler in his 1748 book Introduction to Analysis of the Infin>

It is from this book that we gain the
notation $\pi$ to denote the ratio of a circle’s circumference to its diameter, $e$ to denote the base
of the natural logarithm, the modern notation for the trigonometric functions, as well as the
expression of trigonometric functions as ratios rather than lengths. This book also gives us
Euler’s Identity $(e^{i\pi} + 1 = 0)$ stemming from Euler’s Formula $(e^{i\theta} = cos(\theta) + i \cdot sin(\theta))$, wh>
is itself presented in Introduction to Analysis of the Infinite.

\subsection{Euler's Formula}
$$e^{i \theta}=cos(\theta)+i \cdot sin(\theta)$$
The fact that Euler's Identity uses just one straightforward equation to connect so many crucial mathematical ideas is one >
\footnotetext{\textbf{References for Euler's Formula}}
\footnotetext[1]{Larson, Caleb (2017) An Appreciation of Euler's Formula, Rose-Hulman Undergraduate Mathematics
Journal: Vol. 18 : Iss. 1 , Article 17. }
