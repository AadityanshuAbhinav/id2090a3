\section{ME22B010}
The Drake equation is a probabilistic argument used to estimate the number of active, communicative extraterrestrial civilizations in the Milky Way Galaxy. Frank Drake the creator of the equation created it not for quantifying the number of civilizations but as an opening statement at the first scientific meeting of the Search For Extraterrestrial Intelligence (SETI). The formula is given by 
$$N=R_* \cdot f_p \cdot n_e \cdot f_l \cdot f_i \cdot f_c \cdot L$$
\textit{where,}\\
\textit{$N=number\;of\;civilizations\;with\;which\;humans\;could\;communicate$}\\
\textit{$R_*=mean\;rate\;of\;star\;formation$}\\
\textit{$f_p=fraction\;of\;stars\;that\;have\;planets$}\\
\textit{$n_e=mean\;number\;of\;planets\;that\;could\;support\;life\;per\;star\;with\;planets$}\\
\textit{$f_l=fraction\;of\;life\;supporting\;planets\;that\;develop\;life$}\\
\textit{$f_i=fraction\;of\;planets\;with\;life\;where\;life\;develops\;intelligence$}\\
\textit{$f_c=the\;fraction\;of\;civilizations\;that\;develop\;a\;technology\;that$}\\
\textit{$releases\;detectable\;signs\;of\;their\;existence\;into\;space$}\\
\textit{$L=the\;length\;of\;time\;for\;which\;such\;civilizations\;release\;detectable$}\\
\textit{$signals\;into\;space$}
\\
\\
Theoretical work in the search for extraterrestrial intelligence is dominated by two key concepts: \textbf{The Fermi Paradox} and \textbf{Drake Equation}. The initial answer to the Drake equation was found to be between 1000 and 100,000,000 but this violated Fermi Paradox which was put in layman's terms by Enrico Fermi as "If life is so easy, someone from somewhere must have come calling by now." The current estimate for the parameters of the Drake Equation is $9.1 \times 10^{-13}$ suggesting that we are probably alone in this galaxy and possibly in the observable universe.
\footnote{Reference from https://www.seti.org/drake-equation-index}\\
\\
\emph{Name: Shrijith Belagur} \\
\emph{GithubID: ShrijithBelagur} \\
