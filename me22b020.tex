${\displaystyle mr\omega ^{2}=G{\frac {mM}{r^{2}}}}$
${\displaystyle mr\left({\frac {2\pi }{T}}\right)^{2}=G{\frac {mM}{r^{2}}}\rightarrow T^{2}=\left({\frac {4\pi ^{2}}{GM}
}\right)r^{3}\rightarrow T^{2}\propto r^{3}}$
%\footnote[2]{"https://en.wikipedia.org/wiki/Kepler\%27s_laws_of_planetary_motion"}

The ratio of the square of an object's orbital period with the cube of the semi-major axis of its orbit is the same for
all objects orbiting the same primary.
This captures the relationship between the distance of planets from the Sun, and their orbital periods.

Kepler enunciated in 1619 this third law in a laborious attempt to determine what he viewed as the "music of the spheres
" according to precise laws, and express it in terms of musical notation. It was therefore known as the harmonic law.Usi
ng Newton's law of gravitation (published 1687), this relation can be found in the case of a circular orbit by setting t
he centripetal force equal to the gravitational force. Then, expressing the angular velocity $\Omega$ in terms of the or
bital period we get T and then rearranging, results in Kepler's Third Law.A more detailed derivation can be done with ge
neral elliptical orbits, instead of circles, as well as orbiting the center of mass, instead of just the large mass. Thi
s results in replacing a circular radius with the semi-major axis of the elliptical relative motion of one mass relative
 to the other.In this more rigorous form it is useful for calculation of the orbital period of moons or other binary orb
its like those of binary stars.\footnote[1]{https://www.physicsclassroom.com/class/circles/Lesson-4/Kepler-s-Three-Laws}

However, with planet masses being so much smaller than the Sun, this correction is often ignored.
