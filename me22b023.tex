\section{me22b023}
\textbf{Euler's Totient Function}
\[\phi(n) = n\left(1 - \frac{1}{p_1}\right)\left(1 - \frac{1}{p_2}\right)\ldots\left(1 - \frac{1}{p_k}\right)\]
which can be written as:
\[\phi(n) = n\prod_{p|n}\left(1 - \frac{1}{p}\right)\]
where $p_1,p_2,\ldots,p_n$ are distinct prime factors of $n$ \\
Euler's totient function (also called the Phi function) counts the number of positive integers less than $n$ that are coprime to $n$. $\phi(n)$ is the number of $m \in \mathbf{N}$ such that $1 \le m \le n$ and gcd($m$,$n$) = 1. \\
Euler's Totient Function is a multiplicative function, that is, $\phi(mn) = \phi(m)\phi(n)$ \\
Euler's Totient Function is an important function in number theory. It has applications in cyclotomy and the RSA cryptosystem.
\footnote{
References: 
https://brilliant.org/wiki/eulers-theorem/ \\
https://en.wikipedia.org/wiki/Euler \\
https://www.johndcook.com/blog/2018/09/23/eulers-theorem/ \\
https://www.wallstreetmojo.com/eulers-totient-function/
} \\ 
\textbf{Properties of Euler's Totient Function:}
\begin{itemize}
\item $\phi$ is the symbol used to denote the function.
\item The function deals with the prime numbers theory.
\item The function is applicable only in the case of positive integers.
\item For $\phi(n)$, one can find two multiplicative prime numbers to calculate the function.
\item If integer $n$ is a prime number, then gcd($m$, $n$) = 1.
\item The function works on the formula $1 \le m \le n$, where $m$ and $n$ are the prime and multiplicative numbers.
\item In general, the equation is: 
$\phi(mn) = \phi(m)\phi(n)\left(1 - \frac{1}{m}\right)\left(1 - \frac{1}{m}\right)$

\end{itemize}

Name: Malhar Kamalapur \\
Github user-id: 124229330
