\section*{me22b051}
In this article I will describe an important \textit{quantum field theory} equation, namely the \textbf{Callan–Symanzik equation} \footnote[1]{Callan, Curtis G. (1970-10-15). "Broken Scale Invariance in Scalar Field Theory". Physical Review D. American Physical Society (APS)} given by:

\begin{equation}
\left[  M\frac{\partial }{\partial M} + \beta\left( g \right)\frac{\partial }{\partial g}+n\gamma  \right]G^{\left(n\right)} \left( x_{1},x_{2},...,x_{n};M,g\right)=0 
\end{equation}



where,

\begin{eqnarray}
\beta=\frac{M}{\delta M} \delta g 
\\ \gamma=-\frac{M}{\delta M} \delta \eta
\end{eqnarray}
The \textbf{Callan–Symanzik equation} \footnote[2]{Symanzik, K. (1971). "Small-distance-behaviour analysis and Wilson expansions". Communications in Mathematical Physics. Springer Science and Business Media LLC.} is a differential equation describing the evolution of the n-point correlation functions under variation of the energy scale at which the theory is defined. It involves the beta function of the theory and the anomalous dimensions. \textbf{M} denotes the mass scale and \textbf{g} denotes the renormalised coupling constant in the equation.

The equation takes a different form in \textbf{quantum electrodynamics}, namely:
\begin{equation}
\left[  M\frac{\partial }{\partial M} + \beta\left( e \right)\frac{\partial }{\partial e}+n\gamma_2+m\gamma_3  \right]G^{\left(n,m\right)} \left( x_{1},x_{2},...,x_{n};y_{1},y_{2},...,y_{m};M,e\right)=0 
\end{equation}

where n and m are the numbers of electron and photon fields.The renormalised coupling constant is now the renormalised elementary charge e. The electron field and the photon field rescale differently under renormalisation, and thus lead to two separate functions,$\gamma_{2}$ and $\gamma_{3}$, respectively.

This equation was discovered independently by Curtis Callan and Kurt Symanzik in 1970. 

\section*{Student Information:}
\begin{itemize}
    \item Name: B. SHABARISH 
    \item GitHub User-Id: shabarish169
\end{itemize}
