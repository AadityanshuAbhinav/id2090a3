\section*{me22b099}
Name: Aishani Pattnaik\\
Github User-ID: AishaniPattnaik\\
\subsection*{\underline{The Arrhenius Equation}}
The Arrhenius Equation is a mathematical expression that describes the effect of temperature on the rate of a reaction.
\begin{equation}
k=Ae^{\frac{-E\textsubscript{a}}{RT}}
\end{equation}


\subsection*{Development}
The Arrhenius Equation was proposed by Svante Arrhenius in 1889, based on the work of Dutch chemist Jacobus Henricus Van't Hoff. 
Chemical reactions occur more rapidly at higher temperatures. As the temperature rises, molecules move faster and collide more vigorously, greatly increasing the likelihood of bond cleavages and rearrangements.  
In 1989, Arrhenius finally developed an empirical relation relating the rate of a reaction to its temperature.

\subsection*{Explanation}
In this equation,\\
k is the reaction-rate constant\\
A denotes the pre-exponential factor\\
$E\textsubscript{a}$ is the activation energy for the reaction\\
R is the ideal gas constant (8.314 joules per kelvin per mole) and \\
T is the absolute temperature.

\subsection*{Exponent Term}
RT is the average kinetic energy, this implies that the exponent is just the ratio of the activation energy $E\textsubscript{a}$ to the average kinetic energy. The larger this ratio, the smaller the rate. This means that high temperature and low activation energy favor larger rate constants, and thus speed up the reaction.

\subsection*{Pre-Exponential Term}
This expresses the fraction of reactant molecules that possess enough kinetic energy to react, as governed by the Maxwell-Boltzmann law.\\
The pre-exponential factor is often represented by the following equation
\begin{equation}
A = \rho Z
\end{equation}
Where Z is the frequency factor (frequency of collisions) and $\rho$ is the steric factor (deals with orientation of molecules).
The value of A is determined experimentally since it changes for different reactions.\\

Reference\footnote{R.C Mukerjee. Modern Approach to Chemical Calculations, 1989}
Reference\footnote{https://chem.libretexts.org}
