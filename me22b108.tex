\section{me22b108}
My preferred equation is the Ohm's Law which is:
\begin{equation}
    V=IR
\end{equation}

Ohm’s law, is the description of the relationship between current, voltage, and resistance. The amount of steady current through a large number of materials is directly proportional to the potential difference, or voltage, across the materials.
Thus, if the voltage V (in units of volts) between two ends of a wire made from one of these materials is tripled, the current I (amperes) also triples; and the quotient V/I remains constant. The quotient V/I for a given piece of material is called its resistance, R, measured in units named ohms. The resistance of materials for which Ohm’s law is valid does not change over enormous ranges of voltage and current.
With modifications, Ohm’s law also applies to alternating-current circuits, in which the relation between the voltage and the current is more complicated than for direct currents. Precisely because the current is varying, besides resistance, other forms of opposition to the current arise, called reactance. 

BY: ARYA BHANDARKAR
ROLL NUMBER:ME22B108
GITHUB ID:aryab04

