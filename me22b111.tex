\documentclass{article}
\usepackage[utf8]{inputenc}

\title{me22b111.tex}
\author{Ashwin Kilingar}
\date{04 February 2023}

\begin{document}

\maketitle

\section{Lorentz Force Equation ME22B111}
Let us suppose that there is a point charge q, moving
with a velocity $\vec{v}$ and, located at $\vec{r}$ at a given time t in
presence of both the electric field $\vec{E(r)}$ and the magnetic
field $\vec{B(r)}$. The force on an electric charge q due to both of
them can be written as
\begin{equation}
    \bf\vec{F}\footnote{Referred NCERT Class 12 Physics,Chapter 4:Moving Charges and Magnetic Fields, Section 4.2.2}=q[\vec{E(r)}+\vec{(r)}\times\vec{B}]
\end{equation}

    If we look at the interaction with the magnetic field, we find the following
features
\begin{itemize}
\item {It depends on charge of the particle, the velocity and the
magnetic field. Force on a negative charge is opposite to that on a
positive charge}
\item {The magnetic force includes a vector product of velocity
and magnetic field. The vector product makes the force due to magnetic field vanish (become zero) if velocity and magnetic field are parallel
or anti-parallel. The force acts in a direction perpendicular
to both the velocity and the magnetic field.
Its direction is given by the screw rule or
right hand rule for vector product.}

\item{ The magnetic force is zero if charge is not
moving as $|v|=0$. Only  moving
charge feels the magnetic force}
\end{itemize}
github id: Ash4947

\end{document}

