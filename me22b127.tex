\section{ME22B127}
\subsection{Introduction: De-Broglie Hypothesis}
By the end of the 19th century, only light was thought to have wave particle dual nature and and matter was thought to be of particle nature only. In 1924, De Broglie hypothised that not only light, but all matter in universe has wave nature as well as particle nature. It was known as matter waves. His relation\footnote{Research on the Theory of the Quanta - Louis de Broglie} is given by:
\begin{equation}
\label{rel}
        p\lambda= h
\end{equation}\\
 \ $\lambda$ $\rightarrow$ {Wavelength of the wave}\\
 \ h $\rightarrow$ Planck's constant\\
 \  p $\rightarrow$ Momentum of Particle\\
 This was a major breakthrough in our understanding of Nature.
  \subsection{Proof}
He was able to derive\footnote{ taken from https://en.wikipedia.org/wiki/Matter\_wave\#De\_Broglie\_hypothesis} equation~\ref{rel} by equating energy, assuming matter as particle(from famous Einstein's mass energy equivalence, $E=mc^2$) and assuming it as  a wave ($E={h \nu}$, Max-Planck quantisation theory).\begin{equation}
mc^2=h\nu
\end{equation}

\subsection{Application}
\begin{itemize}
\item This is relation gives the proof to Bohr's quantization of angular momenta. \item This relation is also used to obtain the famous Schrodinger wave eqaution, $\hat H \Psi=E\Psi$.
\end{itemize}
 Name: Gokul R \\
\ Github id: gokulnotre5 \\
