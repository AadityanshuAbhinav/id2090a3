\section{ME22B168}
Name - Nemana Balaseshan
Github id - balaseshan1

\section{Stationary action principle}

\begin{equation}
    \delta S = 0
\end{equation}

where,

\begin{equation}
    S(t_1, t_2, \mathbf{q}) = \int_{t_1}^{t_2}{L(\mathbf{q}, \mathbf{\dot{q}}, t)}
\end{equation}

The stationary-action principle is a variational principle that yields the equations of motion for a physical system. Here S is called the "action" of the physical system. L is the Lagrangian of the system, which is the difference between the kinetic and potential energies of the system. $\mathbf{q}$ is the generalized coordinate of the system and indicates the configuration of the system. The principle can be used to derive Newtonian, Lagrangian and Hamiltonian equations of motion, and even general relativity. In relativity, a different action must be minimized or maximized. Scholars often credit Pierre Louis Maupertuis for formulating the principle of least action because he wrote about it in 1744 and 1746. However, Leonhard Euler discussed the principle in 1744, and evidence shows that Gottfried Leibniz preceded both by 39 years.

\footnote{https://en.wikipedia.org/wiki/Stationary-action\_principle}
