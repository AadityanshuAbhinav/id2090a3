\section{me22b172}
\begin{center}
    {\Large COEFFICIENT OF LINEAR EXPANSION}
\end{center}
\[\Delta l = \alpha \cdot l \cdot T\]
\begin{itemize}
    \item This is used to find the expansion in length a solid is heated
    \item Here $\alpha$ is called linear coefficient of expansion
    \item $\delta$l is change in length
    \item l is original length of the solid 
    \item T is temperature of the solid 
\end{itemize}
$\alpha$ is usually assumed to be independent of temperature. When a solid is heated, it increases in volume. It increases in length, breadth, and thickness. The coefficient of linear expansion may be defined as the increase in length per unit length when the temperature is raised 1°C. The value of the coefficient of expansion varies from substance to substance. The increase in area with temperature, that is, the coefficient of superficial expansion, is approximately twice the coefficient of linear expansion. The coefficient of cubic expansion is almost three times the coefficient of linear expansion.\\
Name : Nivedha A\\
Roll no : me22b172\\
Github user-name : Nivedha2101\\
Reference\footnote{https://www.sciencedirect.com/topics/engineering/linear-coefficient-of-expansion}
