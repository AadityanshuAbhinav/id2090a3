\section{ME22B182}
 \textbf{\textit{Green's Theoram}}\footnote{ Riley, K. F.; Hobson, M. P.; Bence, S. J. (2010). Mathematical Methods for Physics and Engineering. Cambridge University Press. ISBN 978-0-521-86153-3.}
In vector calculus, \textbf{\textit{Green's theorem}} relates a line integral around a simple closed curve C to a double integral over the plane region D bounded by C. It is the two-dimensional special case of \textbf{\textit{Stokes' theorem}}.\\
\begin{equation}
$$\oint$$_c(L dx + M dy)= $$\iint$$ (\frac{\partial M}{\partial x} - \frac{\partial L}{\partial y})dx dy
\end{equation}\\
\\where the path of integration along C is anticlockwise.\\
In physics, Green's theorem finds many applications. One is solving two-dimensional flow integrals, stating that the sum of fluid outflowing from a volume is equal to the total outflow summed about an enclosing area. In plane geometry, and in particular, area surveying, Green's theorem can be used to determine the area and centroid of plane figures solely by integrating over the perimeter.

\subsection{Name and Github ID}


    \item Name: PRIYANSHU RANJAN
    \item Github ID: ME22B182

\end{document}
