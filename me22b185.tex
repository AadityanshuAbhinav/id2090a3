




\begin{itemize}
    \item name: Rishi Dalal
    \item roll no: me22b185
    \item github id: me22b185@smail.iitm.ac.in
\end{itemize}

\section*{Van der Waals equation\footnote{refernce from https://en.wikipedia.org/wiki/Van\_der\_Waals\_equation}}
The Van der Waals  equation is mathematically defined as:
$$\left(P+\frac{a}{V_m^2}\right)\left(V_m-b\right)=RT$$
For n moles of gas, it can also be written as
$$\left(P+\frac{an^2}{V^2}\right)\left(V-nb\right)=nRT$$
where:
\begin{itemize}
    \item P is the pressure of real gas
    \item V is the volume occupied by the real gas
    \item $V_m$ is the molar volume of the real gas
    \item T is the absolute temperature of the real gas
    \item R is the universal gas constant
    \item a,b are Van der Waals constants
    \item n is the number of moles of real gas
\end{itemize}
\subsection*{Assumptions}
 The two major assumptions that led to the violation of Ideal Gas Law are:
\begin{enumerate}
    \item The gas molecules are so small that their volume is negligible compared with the volume occupied by the gas.
    \item The gas molecules don't interact. There are no attractive or repulsive forces between them.
\end{enumerate}
\subsection*{Theory \footnote{ATKINS' PHYSICAL CHEMISTRY;EIGHT EDITION;PART1 EQUILIBRIUM;PROPERTIES OF GASES;REAL GASES}}
 The equation is often written in terms of
the molar volume $Vm = \frac{V}{n}$ as:
 $$P=\frac{RT}{V_m-b}-\frac{a}{V_m^2}$$
 The principal features of the van der Waals equation can be summarized as follows
 \begin{enumerate}
     \item Perfect gas isotherms are obtained at high temperatures and large molar volumes. When the temperature is high, RT may be so large that the first term in the equation greatly exceeds the second. Furthermore, if the molar volume is large in the sense $V_m \gg b$, then the denominator $V_m - b \approx Vm$. Under these conditions, the equation
        reduces to $p = RT/V_m$, the perfect gas equation.
        \item Liquids and gases coexist when cohesive and dispersing effects are in balance.
        \item The critical constants are related to the van der Waals coefficients.At the critical temperature $T_c$, $$ \frac{dP}{dV_m}=\frac{d^2P}{dV^2_m}=0$$ which implies that $V_c=3b$; $P_c=\frac{a}{27b^2}$ ; $T_c=\frac{8a}{27Rb}$

 \end{enumerate}

 
 


