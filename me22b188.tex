%\documentclass{article}
%\usepackage[utf8]{inputenc}
%\usepackage{amsmath}
%\usepackage{hyperref}
%\usepackage{fnpct}
%\hypersetup{
 %   colorlinks=true,
  %  linkcolor=blue,
   % filecolor=blue,      
   % urlcolor=blue,
    %}
%
%\title{ID2090 Assignment 3 - ME22B188}
%\author{Sahil Kelkar, ME22B188}
%\date{February 2023}

%\begin{document}
%\maketitle

\section{ME22B188}
\subsection{Formula}
There are two Friedmann Equations-
\begin{equation} 
\label{eqn1}
\frac{\dot{a}^2 + k c^2}{a^2} = \frac{8 \pi G \rho + \Lambda c^2}{3}
\end{equation} 
And
\begin{equation}
\label{eqn2}
\frac{\ddot{a}}{a} = \frac{-4 \pi G}{3} \left(\rho + \frac{3p} {c^2}\right) + \frac{\Lambda c^2}{3}
\end{equation}
Here,
\begin{itemize}
    \item $\Lambda$ is the cosmological constant
    \item $G$ is the universal constant of gravitation
    \item $c$ is the speed of causality
    \item $\rho$ is the volumetric mass density
    \item $p$ is the pressure
    \item $k$ is constant for a particular solution, but may vary from solution to solution
    \item $a$ is the scaling constant
\end{itemize}

\subsection{About}
These equations were developed by the Soviet mathematician and physicist Alexander Friedmann. Friedmann theorised that the universe was expanding, and consequently developed equations to back up his claim.
Thus the Friedmann Equations are a set of equations which govern the expansion of space, assuming a homogeneous and isotropic universe. Friedmann derived these in 1922 using Einstein's field equations of gravitation.
\paragraph{}
The two equations are independent and derived from different equations.
Equation \ref{eqn1} is derived from the 00 component of Einstein's field equations, while \ref{eqn2} is derived using \ref{eqn1} and the trace of Einstein's field Equations.
\paragraph{}
These equations are truly remarkable, as they allow one to predict the "Fate" of the universe. It is possible to predict the rate of expansion of the universe.\footnote{Reference-https://en.wikipedia.org/wiki/Friedmann\_equations}
$^{,}$
\footnote{https://www.forbes.com/sites/startswithabang/2018/04/17/the-most-important-equation-in-the-universe/}

\subsection{Name and github user id}
Sahil Kelkar, ME22B188, Github User ID- 63910043
%\end{document}
