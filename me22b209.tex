\section{ME22B209}
\subsection{Van der Waal's equation}
\begin{equation}
(P+a\frac{n^2}{V^2})(V-nb)=nRT
\end{equation}
Van der Waal's equation is an equation of state which extends the ideal gas law to include the effects of interaction between molecules of a gas, as well as accounting for the finite size of the molecules.
This equation approximates the behavior of real fluids above their critical temperatures and is qualitatively reasonable for their liquid and low-pressure gaseous states at low temperatures.
The equation relates four state variables: the pressure of the fluid p, the total volume of the fluid's container V, the absolute temperature of the system T, and a and b are the Van der Waal's constant.
It fits pressure-volume-temperature data for a real gas better than the ideal gas equation does.
The van der Waals’ equation corresponds to the case that the repulsive interaction between molecules is non-existent until the molecules come into contact. 
Once they come into contact, the energy required to move them still closer together becomes arbitrarily large.
\footnote{Reference: Class11 NCERT Chemistry}
Name: Vaishnavi Vijayaraj
UserID: me22b209
