\section{ME22B214}
\subsection{Introduction}
In mathematics and physics, the heat equation is a certain partial differential equation. The theory of heat equation was first developed by Joseph Fourier in 1822 for the purpose of modeling how a quantity such as heat diffuses through a given region.
\subsection{Heat Equation}
In the special cases of propagation of heat in an isotropic and homogeneous medium in a 3-dimensional space, this equation is

\begin{equation}
    \frac{\partial u}{\partial t} = \alpha\nabla^2u = \alpha \left(\frac{\partial^2}{\partial x}+\frac{\partial^2}{\partial y}+\frac{\partial^2}{\partial z}\right)u = \alpha(u_{xx}+u_{yy}+u_{zz})
    \label {eqn:heat}
\end{equation}
\\
where:
\begin{itemize}
    \item \emph{u=u(x,y,z,t)} is temperature as a function of space and time. 
    \item $\frac{\partial u}{\partial t}$ is the rate of change of temperature at a point over time.
    \item $u_{xx},u_{yy},u_{zz}$ are the second spatial derivatives of temperature in the \emph{x, y,} and \emph{z} directions, respectively.
    \item $\alpha \equiv \frac{k}{c_p\rho}$ is the thermal diffusivity, a material-specific quantity depending on the thermal conductivity \emph{k}, the specific heat capacity $c_p$ , and the mass density $\rho$ .
\end{itemize}\leavevmode
The heat equation\footnote{https://agupubs.onlinelibrary.wiley.com/doi/abs/10.1029/1998RG900006} is a consequence of Fourier's law of conduction.
Name : Yash Purswani
GitHub ID : BracedHornet186
